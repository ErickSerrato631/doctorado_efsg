\documentclass{article}

% Configuración de idioma
\usepackage[spanish]{babel}
\usepackage[utf8]{inputenc}
\usepackage[T1]{fontenc}

% Tamaño de página y márgenes
\usepackage[letterpaper,top=2cm,bottom=2cm,left=3cm,right=3cm,marginparwidth=1.75cm]{geometry}

% Paquetes útiles
\usepackage{amsmath}
\usepackage{booktabs}
\usepackage{graphicx}
\usepackage{float} % ← necesario para usar [H] en tablas/figuras
\usepackage[colorlinks=true, allcolors=blue]{hyperref}


\title{Notas para tesis de doctorado 25-I}
\author{Erick Felipe Serrato Garcia}

\begin{document}
\maketitle

\section{Modelo}

Definimos las variables: $c$ (células cancerígenas), $s$ (células sanas), $i$ (células inmunes). El modelo dinámico se describe mediante:

\begin{equation}
\frac{dc}{dt} = r_c c \left(\frac{c}{a} - 1\right)\left(1 - \frac{c}{k_c}\right) - \alpha c s - \beta c i - \frac{\mu}{2} (\eta i^2 + \gamma s^2)
\label{eqn:cancer_dynamic}
\end{equation}

\begin{equation}
\frac{ds}{dt} = r_s s \left(1 - \frac{s}{k_s}\right) - \gamma c s + \delta s i^2 - \frac{\mu}{2} \alpha c^2
\label{eqn:sano_dynamic}
\end{equation}

\begin{equation}
\frac{di}{dt} = r_i i \left(1 - \frac{i}{k_i}\right) - \eta c i + \delta s^2 i - \frac{\mu}{2} \beta c^2
\label{eqn:inmune_dynamic}
\end{equation}

\subsection*{Parámetros del modelo}

\begin{table}[H]
\centering
\begin{tabular}{lll}
\toprule
\textbf{Parámetro} & \textbf{Valor} & \textbf{Descripción} \\
\midrule
$r_c$ & 5.84 & Tasa de crecimiento de células cancerígenas \\
$r_s$ & 13.12 & Tasa de crecimiento de células sanas \\
$r_i$ & 10.92 & Tasa de crecimiento de células inmunes \\
$k_c$, $k_s$, $k_i$ & 1.00 & Capacidades de carga de cada población \\
$a$ & 7.22 & Umbral de crecimiento para células cancerígenas \\
$\alpha$ & 10.22 & Inhibición de células sanas por células cancerígenas \\
$\beta$ & 7.60 & Inhibición de células inmunes por células cancerígenas \\
$\gamma$ & 0.74 & Interacción entre células sanas e inmunes \\
$\delta$ & 5.40 & Cooperación entre células sanas e inmunes \\
$\eta$ & 5.08 & Eficiencia inmune contra células cancerígenas \\
$\mu$ & $\in \{0, 1\}$ & Activación binaria de la respuesta inmune \\
$D_c = D_s = D_i$ & 1.00 & Coeficientes de difusión espacial \\
\bottomrule
\end{tabular}
\caption{Valores de los parámetros utilizados en el modelo.}
\end{table}

\section{Resultados: Efecto Allee débil}

\subsection{Estados estacionarios}

Los estados estacionarios se definen como aquellos donde las derivadas temporales se anulan: $\frac{dc}{dt} = \frac{ds}{dt} = \frac{di}{dt} = 0$. Estos puntos representan condiciones de equilibrio entre las tres poblaciones celulares.

\begin{itemize}
\item \textbf{Estable:} El sistema regresa a este punto ante pequeñas perturbaciones. Representa control del tumor.
\item \textbf{Inestable:} Perturbaciones pequeñas pueden llevar al sistema a una proliferación tumoral o colapso inmunológico.
\end{itemize}

\begin{figure}[H]
\centering
\includegraphics[width=0.45\textwidth]{images/steady.png}
\includegraphics[width=0.45\textwidth]{images/steady_mu0.png}
\caption{Estados estacionarios obtenidos para $\mu = 1$ (izquierda) y $\mu = 0$ (derecha). Para $\mu = 1$ se observa un punto inestable $P_1(c,s,i) = (1.0826, 0.4885, 0.1649)$ y uno estable $P_2(c,s,i) = (1.7045, 0.3876, -1.2883)$. Para $\mu = 0$ se observa un punto inestable $P_1(c,s,i) = (1.4103, 1.0414, 0.6110)$ y uno estable $P_2(c,s,i) = (2.9271, 1.6416, -1.6532)$.}
\label{fig:steady_gral}
\end{figure}